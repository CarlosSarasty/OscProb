\hyperlink{namespaceOscProb}{Osc\+Prob} is a small set of classes aimed at computing exact neutrino oscillation probabilities with a few different models.

\hyperlink{namespaceOscProb}{Osc\+Prob} contains a basic framework for computing neutrino oscillation probabilities. It is integrated into \href{https://root.cern.ch/}{\tt R\+O\+OT}, so that each class can be used as you would any R\+O\+OT class.

Available classes are\+:
\begin{DoxyItemize}
\item {\bfseries \hyperlink{classOscProb_1_1PremModel}{Prem\+Model}\+:} Used for determining neutrino paths through the earth
\item {\bfseries \hyperlink{classOscProb_1_1PMNS__Fast}{P\+M\+N\+S\+\_\+\+Fast}\+:} Standard 3-\/flavour oscillations
\item {\bfseries \hyperlink{classOscProb_1_1PMNS__Sterile}{P\+M\+N\+S\+\_\+\+Sterile}\+:} Oscillations with any number of neutrinos
\item {\bfseries \hyperlink{classOscProb_1_1PMNS__NSI}{P\+M\+N\+S\+\_\+\+N\+SI}\+:} Oscillations with 3 flavours including Non-\/\+Standard Interactions
\item {\bfseries \hyperlink{classOscProb_1_1PMNS__Deco}{P\+M\+N\+S\+\_\+\+Deco}\+:} Oscillations with 3 flavours including a simple decoherence model
\end{DoxyItemize}

A few example macros on how to use \hyperlink{namespaceOscProb}{Osc\+Prob} are available in a tutorial directory.

\section*{Installing \hyperlink{namespaceOscProb}{Osc\+Prob}}

\hyperlink{namespaceOscProb}{Osc\+Prob} is very easy to install. The only requirements is to have R\+O\+OT installed with the G\+SL libraries.

{\bfseries I\+M\+P\+O\+R\+T\+A\+NT\+: \hyperlink{namespaceOscProb}{Osc\+Prob} currently does N\+OT build with R\+O\+OT 6}

Once you have R\+O\+OT setup, simply do\+: 
\begin{DoxyCode}
cd OscProb
make
\end{DoxyCode}


A shared library will be produced\+: {\ttfamily lib\+Osc\+Prob.\+so}

This should take a few seconds and you are all set.

Just load the shared library in your R\+O\+OT macros with\+: 
\begin{DoxyCode}
gSystem->Load(\textcolor{stringliteral}{"/full/path/to/libOscProb.so"});
\end{DoxyCode}


In some cases you may need to explicitly load G\+SL libraries. Just add something like this to your rootlogon, replacing the path to the G\+SL libraries if needed\+: 
\begin{DoxyCode}
gSystem->Load(\textcolor{stringliteral}{"/usr/lib/x86\_64-linux-gnu/libgsl.so"});
gSystem->Load(\textcolor{stringliteral}{"/usr/lib/x86\_64-linux-gnu/libgslcblas.so"});
\end{DoxyCode}


\section*{Tutorial}

In the directory Osc\+Prob/tutorial you will find a few macros with examples using \hyperlink{namespaceOscProb}{Osc\+Prob}.

Two macros are particularly useful\+:
\begin{DoxyItemize}
\item {\ttfamily simple\+Examples.\+C} \+: Contains some short pieces of code on how to perform different tasks.
\item {\ttfamily Make\+Oscillogram.\+C} \+: Runs a full example of how to plot an oscillogram with the P\+R\+EM model.
\end{DoxyItemize}

Additionally, the macro {\ttfamily Set\+Nice\+Style.\+C} will provide simple tools to make your plots look nicer. Feel free to use it anytime you\textquotesingle{}re making plots, even if you\textquotesingle{}re not running \hyperlink{namespaceOscProb}{Osc\+Prob}. This is completely independent of \hyperlink{namespaceOscProb}{Osc\+Prob}. 